\documentclass[utf8]{beamer}
\mode<article> % only for the article version
{
  \usepackage{fullpage}
  \usepackage{hyperref}
}

\setbeamercovered{dynamic}
\mode<presentation>
{
  \usetheme{Warsaw}
  %\usecolortheme{beaver}
  \usefonttheme[onlysmall]{structurebold}
  \setbeamertemplate{background canvas}[vertical shading][bottom=red!10,top=blue!10]
  \setbeamercovered{invisible}
}

\usepackage[T1]{fontenc}
%\usepackage[utf8x]{inputenc}
\usepackage{listings}
\usepackage{beamerthemesplit}
\usepackage{wasysym,amssymb,amsmath}

% TIKZ ------------------------------------------------------------------
\usepackage{tikz}
\usetikzlibrary{shapes.geometric}
\usetikzlibrary{fit}
\usetikzlibrary{positioning}
\usetikzlibrary{matrix}
\usetikzlibrary{decorations.pathreplacing}

% THEOREM Environments ---------------------------------------------------
\newtheorem{thm}{Theorem}[section]
\newtheorem{cor}[thm]{Corollary}
\newtheorem{lem}[thm]{Lemma}
\newtheorem{proposition}[thm]{Proposition}
\newtheorem{conj}[thm]{Conjecture}
\newtheorem{prob}[thm]{Problem}
\newtheorem{clm}[thm]{Claim}
\theoremstyle{definition}
\newtheorem{defn}[thm]{Definition}
\theoremstyle{remark}
\newtheorem{rem}[thm]{Remark}

\newcommand{\cExp}[3]{\mathbf{E}_{#1}\left[\left.#2\;\right\vert #3\right]}
\newcommand{\sExp}[2]{\mathbf{E}_{#1}\left[#2\right]}
\newcommand{\Exp}[1]{\sExp{}{#1}}
\newcommand{\pr}[1]{\mathbb{P}\left(#1\right)}
\newcommand{\cProb}[3]{\Pr_{#1}\left[ \left. #2 \;\right\vert #3 \right]}
\newcommand{\sProb}[2]{\Pr_{#1}\left[ #2 ]}

\newcommand{\mA}{\mathcal{A}}
\newcommand{\mB}{\mathcal{B}}
\newcommand{\mC}{\mathcal{C}}
\newcommand{\mF}{\mathcal{F}}
\newcommand{\mP}{\mathcal{P}}
\newcommand{\mG}{\mathcal{G}}
\newcommand{\mL}{\mathcal{L}}
\newcommand{\mX}{\mathcal{X}}
\newcommand{\mD}{\mathcal{D}}
\newcommand{\mM}{\mathcal{M}}
\newcommand{\mS}{\mathcal{S}}
\newcommand{\hG}{\widehat{G}}
\newcommand{\hH}{\widehat{H}}

\newcommand{\Gnp}[2]{\mG(#1,#2)}
\newcommand{\GnP}[1]{\Gnp{#1}{p}}
\newcommand{\GNp}[1]{\Gnp{n}{#1}}
\newcommand{\GNP}{\GNp{p}}

\newcommand{\GNM}{\GNp{M}}
\newcommand{\Bin}{\textrm{Bin}}
\newcommand{\tG}{\widetilde{G}}
\newcommand{\PerfMatch}{\mathcal{PM}}
\newcommand{\Hamiltonicity}{\mathcal{HAM}}
\newcommand{\EdgeConn}[1]{\mathcal{EC}_{#1}}
\newcommand{\VertConn}[1]{\mathcal{VC}_{#1}}

\newcommand{\Small}{\textsc{Small}}

\newcommand{\sep}{:\,}
\colorlet{redshaded}{red!25!bg}
\colorlet{shaded}{black!25!bg}
\colorlet{shadedshaded}{black!10!bg}
\colorlet{blackshaded}{black!40!bg}

\colorlet{darkred}{red!80!black}
\colorlet{darkblue}{blue!80!black}
\colorlet{darkgreen}{green!80!black}

\usepackage[english]{babel}

\newcommand{\pstexInput}[1]{%
  \begin{picture}(0,0)%
    \epsfbox{#1.pstex}%
  \end{picture}%
  \input{#1.pstex_t}%
  }

\newcommand{\blue}[1]{{\color{blue} #1}}
\newcommand{\red}[1]{{\color{red} #1}}

\title{Decision Trees, Random Forests \& Boosting}
\author{Hagai Barmatz, Peleg Michaeli \& Ran Snitkovsky}


\institute{{\color{blue}Tel-Aviv University}}
\date{\today}


\begin{document}
\setbeamercovered{invisible}

\frame{\titlepage
\begin{center}
\vspace {-0.5 in} Theory \& Implementations
\end{center}
}

%\frame{\tableofcontents}

\lstset{language=R,
        basicstyle=\ttfamily,
        keywordstyle=\color{blue}\ttfamily,
        stringstyle=\color{red}\ttfamily,
        commentstyle=\color{green}\ttfamily,
        morecomment=[l][\color{magenta}]{\#}
}

\section{Random Forests}
\subsection{Data}


\begin{frame}[fragile]
  \frametitle{Getting Data}
    \begin{lstlisting}
## The size of our training data
MAXDATA <- 5000
SHIFT <- 30000

## Reading using 'data.table'
require(data.table)
file <- "data/EUR-USD_BID_1MIN-N63619143.csv"
data <- as.data.frame(fread(file))

## Getting close prices
cl <- data[10]
    \end{lstlisting}
\end{frame}
\begin{frame}[fragile]
  \frametitle{Manipulating Data}
    \begin{lstlisting}
## Turn data into a vector
cl.v <- c(cl)[[1]]

## Filter 0's
cl.nz <- cl.v[cl.v!=0]

## Truncate data
cl.tr <- cl.nz[(SHIFT+1):(SHIFT+MAXDATA)]

## Outputs:
## num [1:5000] 1.3 1.3 1.3 1.3 1.3 ...
    \end{lstlisting}
\end{frame}


\subsection{Model}
\begin{frame}
  \frametitle{Suggested Model}
  \begin{block}{Main Idea}
    We're trying to predict the direction of the next close price (1 for
    higher, -1 for lower, 0 for equal), based on the differences between the
    current close price and each of the 100 last close prices.

    This is a \emph{Classification Problem}.
  \end{block}
\end{frame}


\begin{frame}[fragile]
  \frametitle{Learning Data}
    \begin{lstlisting}
## Producing the case matrix
n <- length(cl.tr)
m <- 100
x <- array(dim=c(n,m))
for (i in 1:m) {
  x[,i] <- c(rep(NA, i), diff(cl.tr, lag=i))
}

## Truncating the matrix to avoid NA's
x <- x[(m+1):(n-1),]
    \end{lstlisting}
\end{frame}

\begin{frame}[fragile]
  \frametitle{Response Vector}
    \begin{lstlisting}
## Producing the response vector
diffs <- diff(cl.tr, lag=1)[(m+1):(n-1)]
diffs.sgn <- sign(diffs)

## we use 'factor' to force classification
y <- factor(diffs.sgn)
    \end{lstlisting}
\end{frame}

\begin{frame}[fragile]
  \frametitle{R Package}
  \begin{block}{{\tt randomForest} -- from the docs}
    {\tt randomForest} implements Breiman’s random forest algorithm (based on
    Breiman and Cutler’s original Fortran code) for classification and
    regression. It can also be used in unsupervised mode for assessing
    proximities among data points.
  \end{block}
\end{frame}

\begin{frame}[fragile]
  \frametitle{R Package}
  \begin{block}{Arguments (selection)}
    \begin{description}
      \item[{\tt x}] a data frame or a matrix of predictors describing the
        model to be fitted.
      \item[{\tt y}] A response vector. If a factor, classification is
        assumed, otherwise regression is assumed. If omitted, randomForest
        will run in unsupervised mode.
      \item[{\tt ntree}] Number of trees to grow. This should not be set to
        too small a number, to ensure that every input row gets predicted at
        least a few times.
      \item[{\tt mtry}] Number of variables randomly sampled as candidates at
        each split. Note that the default values are different for
        classification ($\sqrt{p}$ where $p$ is number of variables in {\tt
        x}) and regression ($\frac{p}{3}$)
    \end{description}
  \end{block}
\end{frame}

\begin{frame}[fragile]
  \frametitle{The Model}
    \begin{lstlisting}
## Getting the package
require('randomForest')

## Constructing the model
## We set 'importance=TRUE' to asses importance
## of predictors
rf <- randomForest(x, y, importance=TRUE)

## This takes some time...
    \end{lstlisting}
\end{frame}

\subsection{Output}
\begin{frame}[fragile]
  \frametitle{Model Output}
    \begin{lstlisting}
Call:
 randomForest(x = x, y = y, importance = TRUE) 
     Type of random forest: classification
                     Number of trees: 500
No. of variables tried at each split: 10

        OOB estimate of  error rate: 50.99%
Confusion matrix:
    -1    0   1 class.error
-1 645  300 593   0.5806242
0  350 1124 345   0.3820781
1  587  323 632   0.5901427
    \end{lstlisting}
\end{frame}
\begin{frame}[fragile]
  \frametitle{Model Output}
    \begin{lstlisting}
    plot(rf$importance[,4])
    \end{lstlisting}
    \begin{figure}
      \centering
      \includegraphics[scale=.35]{imp.pdf}
      \label{fig:digraph}
    \end{figure}
\end{frame}

\begin{frame}[fragile]
  \frametitle{Validating Importance}
  \begin{block}{Does it mean anything?}
    \begin{itemize}\pause
      \item All predictors are of low importance\pause
      \item More actual predictors are more important\pause
      \item Is this random? We can check that!
    \end{itemize}
  \end{block}
\end{frame}

\begin{frame}[fragile]
  \frametitle{Random Predictor}
    \begin{lstlisting}
## Producing the new case matrix
x.r <- array(dim=c(n,(m+1)))
for (i in 1:m) {
  x.r[,i] <- c(rep(NA, i), diff(cl.tr, lag=i))
}
x.r[,m+1] <- runif(n, -0.001, 0.001)

## Truncating the matrix to avoid NA's
x.r <- x.r[(m+1):(n-1),]

## Constructing the new model
rf.r <- randomForest(x.r, y, importance=TRUE)
    \end{lstlisting}
\end{frame}

\begin{frame}[fragile]
  \frametitle{Random Predictor -- Output}
    \begin{lstlisting}
    plot(rf.r$importance[,4])
    \end{lstlisting}
    \begin{figure}
      \centering
      \includegraphics[scale=.35]{impr.pdf}
      \label{fig:digraph}
    \end{figure}
\end{frame}

\begin{frame}[fragile]
  \frametitle{Validating Importance}
  \begin{block}{Does it mean anything? (cont.)}
    Will it be smart enough to recognize the importance of a mega-important
    predictor, such as a ... predictor from the future?
  \end{block}
\end{frame}

\begin{frame}[fragile]
  \frametitle{Predictor from the Future}
    \begin{lstlisting}
## Producing the new case matrix
x.f <- array(dim=c(n,(m+1)))
for (i in 1:m) {
  x.f[,i] <- c(rep(NA, i), diff(cl.tr, lag=i))
}

## Truncating the matrix to avoid NA's
x.f <- x.f[(m+1):(n-1),]
x.f[,m+1] <- diffs

## Constructing the new model
rf.f <- randomForest(x.f, y, importance=TRUE)
    \end{lstlisting}
\end{frame}

\begin{frame}[fragile]
  \frametitle{Predictor from the Future -- Output}
    \begin{lstlisting}
Call:
 randomForest(x = x.f, y = y, importance = TRUE) 
     Type of random forest: classification
                     Number of trees: 500
No. of variables tried at each split: 10

        OOB estimate of  error rate: 0.69%
Confusion matrix:
     -1    0    1 class.error
-1 1538    0    0  0.00000000
0    16 1785   18  0.01869159
1     0    0 1542  0.00000000
    \end{lstlisting}
\end{frame}

\begin{frame}[fragile]
  \frametitle{Predictor from the Future -- Output}
    \begin{lstlisting}
    plot(rf.f$importance[,4])
    \end{lstlisting}
    \begin{figure}
      \centering
      \includegraphics[scale=.35]{impf.pdf}
      \label{fig:digraph}
    \end{figure}
\end{frame}

\subsection{Test}
\begin{frame}[fragile]
  \frametitle{Testing the Model}
    \begin{lstlisting}
## The shift of our test data
SHIFT.T <- 60000

## Truncate test data
cl.tst <- cl.nz[(SHIFT.T+1):(SHIFT.T+MAXDATA)]
x.tst <- array(dim=c(n,m))
for (i in 1:m) {
  x.tst[,i] <- c(rep(NA, i), diff(cl.tst, lag=i))
}
x.tst <- x.tst[(m+1):(n-1),]
    \end{lstlisting}
\end{frame}

\begin{frame}[fragile]
  \frametitle{Testing the Model}
    \begin{lstlisting}
## Test response vector
diffs.tst <- diff(cl.tst, lag=1)[(m+1):(n-1)]
diffs.tst.sgn <- sign(diffs.tst)
y.tst <- factor(diffs.tst.sgn)

## Constructing the model
rf.tst <- randomForest(x, y, importance=TRUE,
                       xtest=x.tst, ytest=y.tst)
    \end{lstlisting}
\end{frame}

\begin{frame}[fragile]
  \frametitle{Model Output}
    \begin{lstlisting}
        OOB estimate of  error rate: 51.01%
Confusion matrix:
    -1    0   1 class.error
-1 621  314 603   0.5962289
0  346 1128 345   0.3798791
1  592  299 651   0.5778210
                Test set error rate: 42.19%
Confusion matrix:
    -1    0   1 class.error
-1 556  125 579   0.5587302
0  310 1670 433   0.3079155
1  517  103 606   0.5057096
    \end{lstlisting}
\end{frame}

\section{AdaBoost}
\subsection{Model}
\begin{frame}
  \frametitle{R Package}
  \begin{block}{{\tt adabag} ({\tt boosting}) -- from the docs}
    Fits the AdaBoost.M1 (Freund and Schapire, 1996) and SAMME (Zhu et al.,
    2009) algorithms using classification trees as single classifiers.
  \end{block}\pause
  \begin{block}{{\tt adabag} ({\tt predict.boosting}) -- from the docs}
    Classifies a dataframe using a fitted boosting object.
  \end{block}
\end{frame}

\begin{frame}[fragile]
  \frametitle{The Model}
    \begin{lstlisting}
## Getting the package
require('adabag')

## Constructing the model
x.d <- as.data.frame(x)
x.d$y <- y
boos <- boosting(y~., x.d)
    \end{lstlisting}
\end{frame}

\subsection{Test}
\begin{frame}[fragile]
  \frametitle{Testing the Model}
    \begin{lstlisting}
## Truncate test data
x.d.tst = as.data.frame(x.tst)
x.d.tst$y <- y.tst

## Testing
boos.pr <- predict.boosting(boos, x.d.tst)
    \end{lstlisting}
\end{frame}


\subsection{Output}
\begin{frame}[fragile]
  \frametitle{Model Output}
    \begin{lstlisting}
$confusion
               Observed Class
Predicted Class   -1    0    1
             0    49 1559   32
             1   654  527  698
             -1  557  327  496

$error
[1] 0.4255971

    \end{lstlisting}
\end{frame}

\begin{frame}
  \Huge Thank You!\\ \ \\ $\smiley$
\end{frame}

\end{document}

